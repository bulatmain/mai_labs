\documentclass[a4paper]{article}


\usepackage[T1, T2A]{fontenc}
\usepackage[utf8]{inputenc}
\usepackage[english, russian]{babel}
\usepackage{tempora}
\usepackage[11pt]{extsizes}
\usepackage[top=2cm, bottom=2cm, left=1.5cm, right=1.5cm]{geometry}
\usepackage{enumitem}
\usepackage{setspace}
\usepackage{ifthen}
\usepackage[normalem]{ulem}
\useunder{\uline}{\ulined}{}

\usepackage{tikz}
\usetikzlibrary{graphs, babel, quotes, calc, arrows.meta}

\usepackage{graphicx}

\pagestyle{empty}
\makeatletter
\newcounter{i}
\newcommand{\PrintEmptyLines}[1]{\setcounter{i}{1}\@whilenum\value{i}<#1\do{\stepcounter{i}\EmptyLine\\} \ifnum #1>0 {\EmptyLine}\fi}
\makeatother
\newcommand\arulefill[1]{{\expandafter \ulined #1 \hfill}}
\newcommand{\EmptyLine}{\null\arulefill{}}
\newcommand{\mrule}[1]{\rule[-2.8pt]{#1}{.4pt}}
\newcounter{EmptyLines1}
\newcounter{EmptyLines2}
\newcounter{EmptyLines3}
\newcounter{EmptyLines10}
\newcounter{EmptyLines11}
\newcounter{EmptyLines12}

\linespread{0.9}

%%%%%%%%%%%%%%%%%%%%%%%%%%%%%%
%% НИЖЕ ПОЛЯ ДЛЯ ЗАПОЛНЕНИЯ %%
%%%%%%%%%%%%%%%%%%%%%%%%%%%%%%

\newcommand{\LabNumber}{IX}  % номер ЛР
\newcommand{\Discipline}{Алгоритмы и структуры данных}  % название дисциплины (курса)
\newcommand{\StudentGroup}{М8О-103Б-22}  % группа студента
\newcommand{\StudentName}{Ахметшин Булат Рамилевич}  % имя студента
\newcommand{\StudentNumber}{2}  % номер по списку
\newcommand{\StudentContacts}{ahmbulat04@yandex.ru}  % контакты студента
\newcommand{\DateOfCompletion}{05.05.2023 г.}  % дата выполнения работы
\newcommand{\Teacher}{доцент каф. 806 Никулин С.П.}  % преподаватель
\newcommand{\NumberVariant}{2 - \{2, 8\}}  % номер варианта

\newcommand{\Texti}{}  % Тема
\newcommand{\Textii}{
}  % Цель работы
\newcommand{\Textiii}{
}  % Задание 
\newcommand{\Textvi}{
}  % Идея, метод, алгоритм
\newcommand{\Textvii}{
}  % Сценарий выполнения работы

\newcommand{\Textviii}{\scriptsize }  % Распечатка протокола

\newcommand{\Textx}{}  % Замечания автора
\newcommand{\Textxi}{
}  % Выводы
\newcommand{\Textxii}{}  % Устранение недочётов

% Количество пустых подчеркнутых полей для:

\setcounter{EmptyLines1}{1}  % Тема
\setcounter{EmptyLines2}{2}  % Цель работы
\setcounter{EmptyLines3}{3}  % Задание
\setcounter{EmptyLines10}{3}  % Замечания автора
\setcounter{EmptyLines11}{3}  % Выводы
\setcounter{EmptyLines12}{5}  % Устранение недочётов

\begin{document}
\begin{minipage}{0.1\textwidth}
    \fbox{\rule[2.1cm]{1.7cm}{0pt}}
    \vspace{3.5cm}
\end{minipage}
\begin{minipage}{0.858\textwidth}
    \begin{center}
        \Large\textbf{Отчет по лабораторной работе № {\LabNumber} по курсу \uline\Discipline}
    \end{center}
    
    \begin{doublespace}
        \vbox{\hfill\vbox{%
        \hbox{Студент группы \uline{\StudentGroup \ \StudentName}, № по списку \uline\StudentNumber}%
        \hbox{Контакты www, e-mail, icq, skype \uline{\StudentContacts}}
        }}
    \end{doublespace}
    \begin{doublespace}
        \vbox{\hfill\vbox{%
        \hbox{Работа выполнена: \DateOfCompletion}%
        \hbox{Преподаватель: \uline{\Teacher}}%
        \hbox{Входной контроль знаний с оценкой \mrule{3.3cm}}%
        \hbox{Отчет сдан << \hspace{0.3cm} >> \mrule{1.5cm} 202 \mrule{.3cm} г., итоговая оценка \mrule{.5cm}}
        }}
    \end{doublespace}
    \hfill\hbox{Подпись преподавателя \mrule{3cm}}
\end{minipage}


\begin{enumerate}[label=\textbf{\arabic*}.]

\item \textbf{Тема:} {\footnotesize Сортировка и поиск.} \arulefill{\Texti} \\ 
\PrintEmptyLines{\value{EmptyLines1}}

\item \textbf{Цель работы:} {\footnotesize Составить программу на языке Си с использованием процедур и функций для сортировки таблицы заданным методом и двоичного поиска по ключу в таблице.} \arulefill{\Textii} \\
\PrintEmptyLines{\value{EmptyLines2}}

\item \textbf{Задание} (\textit{вариант № \NumberVariant}): {\footnotesize Сортировка - линейный выбор с подсчетом, таблица - ключ из строки и целого, 32 байта на ключ, хранение данных и ключей вместе, минимальное число элементов - 18.} \arulefill{\Textiii} \\
\PrintEmptyLines{\value{EmptyLines3}}

\item \textbf{Оборудование} (\textit{лабораторное}):\\
ЭВМ \EmptyLine, процессор \EmptyLine, имя узла сети \EmptyLine \ c ОП \EmptyLine \ Мб,\\
НМД \EmptyLine \ Мб. Терминал \EmptyLine \ адрес \EmptyLine. Принтер \EmptyLine \\
Другие устройства \EmptyLine \\ \EmptyLine \\

%=========================================================
%====================ОБОРУДОВАНИЕ=========================
%=========================================================
\textit{Оборудование ПЭВМ студента, если использовалось:}\\
Процессор \arulefill{Intel(R) Core(TM) i7-10510U}\EmptyLine \ с ОП \arulefill{8 ГБ}\EmptyLine \, НМД \arulefill{SSD 512 ГБ}\EmptyLine \ . Монитор \arulefill{Встроенный 1920х1080}\EmptyLine \\
Другие устройства \arulefill{}\EmptyLine \\ \EmptyLine
%=========================================================

\item \textbf{Программное обеспечение} (\textit{лабораторное}):\\
Операционная система семейства \EmptyLine, наименование \EmptyLine \ версия \EmptyLine \\
интерпретатор команд \EmptyLine \ версия \EmptyLine \\
Система программирования \EmptyLine \ версия \EmptyLine \\
Редактор текстов \EmptyLine \ версия \EmptyLine \\
Утилиты операционной системы \EmptyLine \\ \EmptyLine \\
Прикладные системы и программы \EmptyLine \\
Местонахождение и имена файлов программ и данных \EmptyLine \\ \EmptyLine \\

%=========================================================
%====================ОБЕСПЕЧЕНИЕ==========================
%=========================================================
\textit{Программное обеспечение ЭВМ студента, если использовалось:}\\
Операционная система семейства \arulefill{UNIX}\EmptyLine, наименование \arulefill{Ubuntu}\EmptyLine \ версия \arulefill{22.04}\EmptyLine \\
интерпретатор команд \arulefill{GNU bash}\EmptyLine \ версия \arulefill{5.1.16}\EmptyLine \\
Система программирования \arulefill{ Visual Studio Code}\EmptyLine \ версия \arulefill{ 1.77.3}\EmptyLine \\
Редактор текстов \arulefill{Sublime Text 3}\EmptyLine \ версия \arulefill{3211}\EmptyLine \\
Утилиты операционной системы \arulefill{Стандартные утилиты OS Linux}\EmptyLine \\
Прикладные системы и программы \arulefill{Редактор текста nano.}\EmptyLine \\
Местонахождение и имена файлов программ и данных на домашнем компьютере \\\arulefill{/home/bulat/Studying/prprm/cr/IX/}\EmptyLine \\
%=========================================================

\item 
\begin{minipage}[t][0.45\textheight]{.95\textwidth}
\textbf{Идея, метод, алгоритм} {\footnotesize решение задачи (в формах: словесной, псевдокода, графической [блок-схема, диаграмма, рисунок, таблица] или формальные спецификации с пред- и постусловиями)} \\ \Textvi

{\footnotesize Для реализации сортировки линейной выборкой будут учитываться только целача часть значения ключей, т.к. сам алгоритм работает с таким типом ключей. \\
В качестве тестовых примеров я возьму список сотрудников некой организации, ключом сотрудника будет департамент и индекс, значением - имя сотрудника.} 
\end{minipage}
\item 
\begin{minipage}[t][0.43\textheight]{.95\textwidth}
\textbf{Сценарий выполнения работы} {\footnotesize (план работы, первоначальный текст программы в черновике [можно на отдельном листе] и тесты либо соображения по тестированию)} \\ \Textvii

{\footnotesize Составить makefile, написать тестовые файлы, релизовать таблицу и функции обработки и сортировки, отладить, составить протокол.}

\end{minipage}

\textit{Пункты 1-7 отчета составляются строго до начала лабораторной работы.} \\
\hfill\hbox{\textit{Допущен к выполнению работы.} \textbf{Подпись преподавателя \mrule{4cm}}}

\item 
\textbf{Распечатка протокола} {\footnotesize (подклеить листинг окончательного варианта программы с тестовыми примерами, подписанный преподавателем)} \\

%=========================================================
%====================ПРОТОКОЛ=============================
%=========================================================

\scriptsize
\centering
\begin{verbatim}
bulat@bulat-Swift-SF314-58:~/Studying/prprm/cr/IV$ script logs/proto1
Script started, output log file is 'logs/proto1'.
bulat@bulat-Swift-SF314-58:~/Studying/prprm/cr/IV$ ls
logs    makefile             sorted.txt  tex
main.c  reversed_sorted.txt  table.h     unsorted.txt
bulat@bulat-Swift-SF314-58:~/Studying/prprm/cr/IV$ cat makefile 
CC = gcc
CFLAGS = -std=c99 -Wall -Wextra -Wformat=0

main:
	$(CC) $(CFLAGS) -o main main.c
debug:
	$(CC) $(CFLAGS) -g -o main main.c

clean:
	rm -f *.o mainbulat@bulat-Swift-SF314-58:~/Studying/prprm/cr/IV$ cat table.h 
#ifndef TABLE_H
#define TABLE_H

#include <inttypes.h>
#include <stdbool.h>
#include <stdio.h>
#include <stdint.h>
#include <string.h>
#include <stdlib.h>

#define STRING_KEY_CAP 28

typedef uint64_t size_t;

typedef struct {
    char key_s[STRING_KEY_CAP];
    int key_int;
} complex_key;

typedef struct {
    complex_key key;
    char *value;
} item;

typedef struct {
    item *rows;
    uint64_t count;
    uint64_t max_key;
} table;


table table_alloc();

table table_copy(table t);
void table_push(table *t, char *key_s, int key_int, char *value);
table table_quick_sort(table t);

char *table_binary_search(table t, char *key_s, int key_int);

void table_print(table t);

void table_dealloc(table *t);


#endif // TABLE_H



#ifdef TABLE_IMPLEMENTATION

table table_alloc() {
    table t;
    t.rows = (item*) calloc(0, sizeof(item));
    t.count = 0;
    t.max_key = 0;
    return t;
}


item item_copy(item* const a) {
    complex_key ck = {
        .key_int = a->key.key_int
    };

    for (int i = 0; i < STRING_KEY_CAP && i < (int) strlen(a->key.key_s); ++i) {
        ck.key_s[i] = a->key.key_s[i];
    }
    
    char *value = (char*) calloc(256, sizeof(char));
    strcpy(value, a->value);

    item i = {
        .key = ck,
        .value = value
    };
    return i;
}

table table_copy(table t) {
    table temp = table_alloc();

    for (uint64_t i = 0; i < t.count; ++i) {
        char *value = (char*) calloc(256, sizeof(char));
        strcpy(value, t.rows[i].value);
        table_push(&temp, t.rows[i].key.key_s, t.rows[i].key.key_int, value);
    }

    return temp;
}

void table_push(table *t, char *key_s, int key_int, char *value) {
    complex_key ck = {
        .key_int = key_int
    };

    for (int i = 0; i < STRING_KEY_CAP && i < (int) strlen(key_s); ++i) {
        ck.key_s[i] = key_s[i];
    }

    item i = {
        .key = ck,
        .value = value
    };

    if (t->max_key < key_int) {
        t->max_key = key_int;
    }

    t->rows = (item*) realloc(t->rows, sizeof(item) * (t->count + 1));
    t->count++;

    t->rows[t->count - 1] = i;
}

table counting_sort(table* const A) {
    size_t n = A->count, k = A->max_key;

    size_t* P = (size_t*)calloc(k, sizeof(size_t));

    for (size_t i = 0; i < n; ++i) {
        ++P[A->rows[i].key.key_int - 1];
    }
    for (size_t i = 1; i < k; ++i) {
        P[i] += P[i - 1];
    }
    for (size_t i = k; i > 1; --i) {
        P[i - 1] = P[i - 2];
    }
    P[0] = 0;

    table B;
    B.rows = (item*)calloc(n, sizeof(item));

    for (size_t i = 0; i < n; ++i) {
        size_t p = A->rows[i].key.key_int;
        item t = item_copy(&(A->rows[i]));
        B.rows[P[p - 1]] = t;
        ++P[p - 1];
    }

    B.count = A->count;
    B.max_key = A->max_key;

    return B;
}

void _table_quick_sort(table *t, uint64_t l, uint64_t r) {
    char *pivot_value = t->rows[l].value;
    complex_key pivot_key = t->rows[l].key;
    uint64_t l_init = l, r_init = r;

    while (l < r) {
        while (
            (
                strcmp(pivot_key.key_s, t->rows[r].key.key_s) < 0 ||
                (
                    strcmp(pivot_key.key_s, t->rows[r].key.key_s) == 0 &&
                    pivot_key.key_int <= t->rows[r].key.key_int
                )
            ) && (l < r)
        ) r--;

        if (l != r) {
            t->rows[l].value = t->rows[r].value;
            t->rows[l].key = t->rows[r].key;
            l++;
        }

        while (
            strcmp(pivot_key.key_s, t->rows[l].key.key_s) > 0 &&
            (l < r)
        ) l++;

        if (l != r) {
            t->rows[r].value = t->rows[l].value;
            t->rows[r].key = t->rows[l].key;
            r--;
        }
    }

    t->rows[l].key = pivot_key;
    t->rows[l].value = pivot_value;

    uint64_t pivot = l;
    l = l_init;
    r = r_init;

    if (l < pivot) _table_quick_sort(t, l, pivot - 1);
    if (r > pivot) _table_quick_sort(t, pivot + 1, r);
}

// sort keys by elements
table table_quick_sort(table t) {
    table temp = table_copy(t);

    _table_quick_sort(&temp, 0, temp.count - 1);

    return temp;
}


char *table_binary_search(table t, char *key_s, int key_int) {
    int64_t l = 0, r = t.count - 1, m = (l + r)/2;

    while (l <= r) {
        m = (l + r)/2;

        char *curr_key_s = t.rows[m].key.key_s;
        int curr_key_int = t.rows[m].key.key_int;

        if (strcmp(key_s, curr_key_s) == 0 && key_int == curr_key_int) {
            return t.rows[m].value;
        }
 
        if ((key_int >= curr_key_int)) {
            l = m + 1;
        } else {
            r = m - 1;
        }
    }

    return NULL;
}


void table_print(table t) {
    printf(
        "| Key%*s| Value\n", STRING_KEY_CAP + 11 - 3, ""
    );

    for (uint64_t i = 0; i < t.count; ++i) {
        printf("| %*s%11d| %s\n", STRING_KEY_CAP, t.rows[i].key.key_s, t.rows[i].key.key_int, t.rows[i].value);
    }
}


void table_dealloc(table *t) {
    for (uint64_t i = 0; i < t->count; ++i) {
        t->rows[i].key.key_s[0] = '\0';
        t->rows[i].key.key_int = 0;
        free(t->rows[i].value);
    }

    free(t->rows);
}

#endif // TABLE_IMPLEMENTATIONbulat@bulat-Swift-SF314-58:~/Studying/prprm/cr/IV$ cat main.c 
#define TABLE_IMPLEMENTATION
#include "table.h"


int main() {
    table t = table_alloc();

    uint64_t count = 0;
    scanf("%ld", &count);
    
    printf("Table: ");
    for (uint64_t i = 0; i < count; ++i) {

        char *key_s = (char*) calloc(256, sizeof(char));
        scanf("%s", key_s);

        int key_int = 0;
        scanf("%d", &key_int);

        char *value = (char*) calloc(256, sizeof(char));
        scanf("%s", value);

        table_push(&t, key_s, key_int, value);

        free(key_s);
    }
    printf("\n");

    table_print(t);
    printf("\n");
    table s = counting_sort(&t);
    table_print(s);

    int search = 1;
    while (search) {
        char *key_s = (char*) calloc(256, sizeof(char));
        int key_int = 0;
        scanf("%s", key_s);
        scanf("%d", &key_int);
        printf("Search for:(%s, %d)\n", key_s, key_int);

        printf("Found string: %s\n", table_binary_search(s, key_s, key_int));
        free(key_s);

        printf("Do you want to continue the search? (0/1): ");
        scanf("%d", &search);
    }

    printf("\n");

    table_dealloc(&t);
}bulat@bulat-Swift-SF314-58:~/Studying/prprm/cr/IV$ make
gcc -std=c99 -Wall -Wextra -Wformat=0 -o main main.c
In file included from main.c:2:
table.h: In function ‘table_push’:
table.h:105:20: warning: comparison of integer expressions of different signedness: ‘uint64_t’ {aka ‘long unsigned int’} and ‘int’ [-Wsign-compare]
  105 |     if (t->max_key < key_int) {
      |                    ^
bulat@bulat-Swift-SF314-58:~/Studying/prprm/cr/IV$ ./main <sorted.txt 
Table: 
| Key                                    | Value
|     TransportationDepartment          1| Mike
|              LegalDepartment          2| George
|     TransportationDepartment          3| Jacob
|     TransportationDepartment          4| Kevin
|            DevelopmentCentre          5| Alice
|              LegalDepartment          6| Ioan
|            DevelopmentCentre          7| Frida
|            DevelopmentCentre          8| Alice
|          FinancialDepartment          9| Alex
|            DevelopmentCentre         10| Morty
|          FinancialDepartment         11| Rick
|          FinancialDepartment         12| Daniel
|          FinancialDepartment         13| Belle
|          FinancialDepartment         14| Michel
|            DevelopmentCentre         15| Patrix
|     TransportationDepartment         16| Paul
|            DevelopmentCentre         17| Jane
|              LegalDepartment         18| Freya

| Key                                    | Value
|     TransportationDepartment          1| Mike
|              LegalDepartment          2| George
|     TransportationDepartment          3| Jacob
|     TransportationDepartment          4| Kevin
|            DevelopmentCentre          5| Alice
|              LegalDepartment          6| Ioan
|            DevelopmentCentre          7| Frida
|            DevelopmentCentre          8| Alice
|          FinancialDepartment          9| Alex
|            DevelopmentCentre         10| Morty
|          FinancialDepartment         11| Rick
|          FinancialDepartment         12| Daniel
|          FinancialDepartment         13| Belle
|          FinancialDepartment         14| Michel
|            DevelopmentCentre         15| Patrix
|     TransportationDepartment         16| Paul
|            DevelopmentCentre         17| Jane
|              LegalDepartment         18| Freya
Search for:(FinancialDepartment, 11)
Found string: Rick
Do you want to continue the search? (0/1): Search for:(LegalDepartment, 6)
Found string: Ioan
Do you want to continue the search? (0/1): Search for:(TransportationDepartment, 3)
Found string: Jacob
Do you want to continue the search? (0/1): 
bulat@bulat-Swift-SF314-58:~/Studying/prprm/cr/IV$ ./main <unsorted.txt 
Table: 
| Key                                    | Value
|     TransportationDepartment          1| Mike
|     TransportationDepartment          4| Kevin
|            DevelopmentCentre          8| Alice
|              LegalDepartment         18| Freya
|          FinancialDepartment          9| Alex
|            DevelopmentCentre         15| Patrix
|            DevelopmentCentre          5| Alice
|            DevelopmentCentre          7| Frida
|              LegalDepartment          2| George
|     TransportationDepartment         16| Paul
|            DevelopmentCentre         10| Morty
|     TransportationDepartment          3| Jacob
|          FinancialDepartment         13| Belle
|          FinancialDepartment         12| Daniel
|              LegalDepartment          6| Ioan
|          FinancialDepartment         14| Michel
|          FinancialDepartment         11| Rick
|            DevelopmentCentre         17| Jane

| Key                                    | Value
|     TransportationDepartment          1| Mike
|              LegalDepartment          2| George
|     TransportationDepartment          3| Jacob
|     TransportationDepartment          4| Kevin
|            DevelopmentCentre          5| Alice
|              LegalDepartment          6| Ioan
|            DevelopmentCentre          7| Frida
|            DevelopmentCentre          8| Alice
|          FinancialDepartment          9| Alex
|            DevelopmentCentre         10| Morty
|          FinancialDepartment         11| Rick
|          FinancialDepartment         12| Daniel
|          FinancialDepartment         13| Belle
|          FinancialDepartment         14| Michel
|            DevelopmentCentre         15| Patrix
|     TransportationDepartment         16| Paul
|            DevelopmentCentre         17| Jane
|              LegalDepartment         18| Freya
Search for:(FinancialDepartment, 11)
Found string: Rick
Do you want to continue the search? (0/1): Search for:(LegalDepartment, 6)
Found string: Ioan
Do you want to continue the search? (0/1): Search for:(TransportationDepartment, 3)
Found string: Jacob
Do you want to continue the search? (0/1): 
bulat@bulat-Swift-SF314-58:~/Studying/prprm/cr/IV$ ./main <reversed_sorted.txt 
Table: 
| Key                                    | Value
|              LegalDepartment         18| Freya
|            DevelopmentCentre         17| Jane
|     TransportationDepartment         16| Paul
|            DevelopmentCentre         15| Patrix
|          FinancialDepartment         14| Michel
|          FinancialDepartment         13| Belle
|          FinancialDepartment         12| Daniel
|          FinancialDepartment         11| Rick
|            DevelopmentCentre         10| Morty
|          FinancialDepartment          9| Alex
|            DevelopmentCentre          8| Alice
|            DevelopmentCentre          7| Frida
|              LegalDepartment          6| Ioan
|            DevelopmentCentre          5| Alice
|     TransportationDepartment          4| Kevin
|     TransportationDepartment          3| Jacob
|              LegalDepartment          2| George
|     TransportationDepartment          1| Mike

| Key                                    | Value
|     TransportationDepartment          1| Mike
|              LegalDepartment          2| George
|     TransportationDepartment          3| Jacob
|     TransportationDepartment          4| Kevin
|            DevelopmentCentre          5| Alice
|              LegalDepartment          6| Ioan
|            DevelopmentCentre          7| Frida
|            DevelopmentCentre          8| Alice
|          FinancialDepartment          9| Alex
|            DevelopmentCentre         10| Morty
|          FinancialDepartment         11| Rick
|          FinancialDepartment         12| Daniel
|          FinancialDepartment         13| Belle
|          FinancialDepartment         14| Michel
|            DevelopmentCentre         15| Patrix
|     TransportationDepartment         16| Paul
|            DevelopmentCentre         17| Jane
|              LegalDepartment         18| Freya
Search for:(FinancialDepartment, 11)
Found string: Rick
Do you want to continue the search? (0/1): Search for:(LegalDepartment, 6)
Found string: Ioan
Do you want to continue the search? (0/1): Search for:(TransportationDepartment, 3)
Found string: Jacob
Do you want to continue the search? (0/1): 

\end{verbatim}

\large{}

\newpage

\item \textbf{Дневник отладки} {\footnotesize должен содержать дату и время сеансов отладки и основные события (ошибки в сценарии и программе, нестандартные ситуации) и краткие комментарии к ним. В дневнике отладки приводятся сведения об использовании ЭВМ, существенном участии преподавателя и других лиц в написании и отладке программы.} \\
\begin{tabular}[t]{|c|c|c|c|c|c|c|}
\hline
№ & \begin{tabular}[t]{c} Лаб. \\ или \\ дом. \end{tabular} & Дата & Время & \hspace{.7cm} Событие \hspace{.7cm} & Действие по исправлению & \hspace{.7cm} Примечание \hspace{.7cm} \\
\hline
\begin{minipage}[t][0.45\textheight]{0.01\textwidth}\end{minipage} & & & & & &\\
\hline
\end{tabular}

\item \textbf{Замечания автора} {\footnotesize по существу работы:} \arulefill{\Textx} \\
\PrintEmptyLines{\value{EmptyLines2}}\\

\item \textbf{Выводы:} \arulefill{\footnotesize В ходе этой лабораторной работы я получил опыт реализации таблиц с комплесными ключами и функций для их обработки и сортировки.} \Textxi \\
\PrintEmptyLines{\value{EmptyLines2}}\\


\item Недочёты при выполнении задания могут быть устранены следующим образом: {\footnotesize} \arulefill{\Textx}
\PrintEmptyLines{\value{EmptyLines2}}\\

 \begin{flushright}
Подпись студента \mrule{4cm}
\end{flushright}

\end{enumerate}

\end{document}