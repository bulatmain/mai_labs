\documentclass[a4paper]{article}


\usepackage[T1, T2A]{fontenc}
\usepackage[utf8]{inputenc}
\usepackage[english, russian]{babel}
\usepackage{tempora}
\usepackage[11pt]{extsizes}
\usepackage[top=2cm, bottom=2cm, left=1.5cm, right=1.5cm]{geometry}
\usepackage{enumitem}
\usepackage{setspace}
\usepackage{ifthen}
\usepackage[normalem]{ulem}
\useunder{\uline}{\ulined}{}

\usepackage{tikz}
\usetikzlibrary{graphs, babel, quotes, calc, arrows.meta}

\usepackage{graphicx}

\pagestyle{empty}
\makeatletter
\newcounter{i}
\newcommand{\PrintEmptyLines}[1]{\setcounter{i}{1}\@whilenum\value{i}<#1\do{\stepcounter{i}\EmptyLine\\} \ifnum #1>0 {\EmptyLine}\fi}
\makeatother
\newcommand\arulefill[1]{{\expandafter \ulined #1 \hfill}}
\newcommand{\EmptyLine}{\null\arulefill{}}
\newcommand{\mrule}[1]{\rule[-2.8pt]{#1}{.4pt}}
\newcounter{EmptyLines1}
\newcounter{EmptyLines2}
\newcounter{EmptyLines3}
\newcounter{EmptyLines10}
\newcounter{EmptyLines11}
\newcounter{EmptyLines12}

\linespread{0.9}

%%%%%%%%%%%%%%%%%%%%%%%%%%%%%%
%% НИЖЕ ПОЛЯ ДЛЯ ЗАПОЛНЕНИЯ %%
%%%%%%%%%%%%%%%%%%%%%%%%%%%%%%

\newcommand{\LabNumber}{VII}  % номер ЛР
\newcommand{\Discipline}{Алгоритмы и структуры данных}  % название дисциплины (курса)
\newcommand{\StudentGroup}{М8О-103Б-22}  % группа студента
\newcommand{\StudentName}{Ахметшин Булат Рамилевич}  % имя студента
\newcommand{\StudentNumber}{2}  % номер по списку
\newcommand{\StudentContacts}{ahmbulat04@yandex.ru}  % контакты студента
\newcommand{\DateOfCompletion}{05.05.2023 г.}  % дата выполнения работы
\newcommand{\Teacher}{доцент каф. 806 Никулин С.П.}  % преподаватель
\newcommand{\NumberVariant}{2 - \{2, 2, 2\}}  % номер варианта

\newcommand{\Texti}{}  % Тема
\newcommand{\Textii}{
}  % Цель работы
\newcommand{\Textiii}{
}  % Задание 
\newcommand{\Textvi}{
}  % Идея, метод, алгоритм
\newcommand{\Textvii}{
}  % Сценарий выполнения работы

\newcommand{\Textviii}{\scriptsize }  % Распечатка протокола

\newcommand{\Textx}{}  % Замечания автора
\newcommand{\Textxi}{
}  % Выводы
\newcommand{\Textxii}{}  % Устранение недочётов

% Количество пустых подчеркнутых полей для:

\setcounter{EmptyLines1}{1}  % Тема
\setcounter{EmptyLines2}{2}  % Цель работы
\setcounter{EmptyLines3}{3}  % Задание
\setcounter{EmptyLines10}{3}  % Замечания автора
\setcounter{EmptyLines11}{3}  % Выводы
\setcounter{EmptyLines12}{5}  % Устранение недочётов

\begin{document}
\begin{minipage}{0.1\textwidth}
    \fbox{\rule[2.1cm]{1.7cm}{0pt}}
    \vspace{3.5cm}
\end{minipage}
\begin{minipage}{0.858\textwidth}
    \begin{center}
        \Large\textbf{Отчет по лабораторной работе № {\LabNumber} по курсу \uline\Discipline}
    \end{center}
    
    \begin{doublespace}
        \vbox{\hfill\vbox{%
        \hbox{Студент группы \uline{\StudentGroup \ \StudentName}, № по списку \uline\StudentNumber}%
        \hbox{Контакты www, e-mail, icq, skype \uline{\StudentContacts}}
        }}
    \end{doublespace}
    \begin{doublespace}
        \vbox{\hfill\vbox{%
        \hbox{Работа выполнена: \DateOfCompletion}%
        \hbox{Преподаватель: \uline{\Teacher}}%
        \hbox{Входной контроль знаний с оценкой \mrule{3.3cm}}%
        \hbox{Отчет сдан << \hspace{0.3cm} >> \mrule{1.5cm} 202 \mrule{.3cm} г., итоговая оценка \mrule{.5cm}}
        }}
    \end{doublespace}
    \hfill\hbox{Подпись преподавателя \mrule{3cm}}
\end{minipage}


\begin{enumerate}[label=\textbf{\arabic*}.]

\item \textbf{Тема:} {\footnotesize Разреженные матрицы.} \arulefill{\Texti} \\ 
\PrintEmptyLines{\value{EmptyLines1}}

\item \textbf{Цель работы:} {\footnotesize Составить и программу на языке Си для обработки прямоугольных разряженных матриц с элементами вещественного типа.} \arulefill{\Textii} \\
\PrintEmptyLines{\value{EmptyLines2}}

\item \textbf{Задание} (\textit{вариант № \NumberVariant}): {\footnotesize Определить максимальный по модулю элемент матрицы и разделить на него все элементы столбца, в котором он находится. Если таких элементов несколько, обработать предпоследний стобец, содержащий такой элемент.} \arulefill{\Textiii} \\
\PrintEmptyLines{\value{EmptyLines3}}

\item \textbf{Оборудование} (\textit{лабораторное}):\\
ЭВМ \EmptyLine, процессор \EmptyLine, имя узла сети \EmptyLine \ c ОП \EmptyLine \ Мб,\\
НМД \EmptyLine \ Мб. Терминал \EmptyLine \ адрес \EmptyLine. Принтер \EmptyLine \\
Другие устройства \EmptyLine \\ \EmptyLine \\

%=========================================================
%====================ОБОРУДОВАНИЕ=========================
%=========================================================
\textit{Оборудование ПЭВМ студента, если использовалось:}\\
Процессор \arulefill{Intel(R) Core(TM) i7-10510U}\EmptyLine \ с ОП \arulefill{8 ГБ}\EmptyLine \, НМД \arulefill{SSD 512 ГБ}\EmptyLine \ . Монитор \arulefill{Встроенный 1920х1080}\EmptyLine \\
Другие устройства \arulefill{}\EmptyLine \\ \EmptyLine
%=========================================================

\item \textbf{Программное обеспечение} (\textit{лабораторное}):\\
Операционная система семейства \EmptyLine, наименование \EmptyLine \ версия \EmptyLine \\
интерпретатор команд \EmptyLine \ версия \EmptyLine \\
Система программирования \EmptyLine \ версия \EmptyLine \\
Редактор текстов \EmptyLine \ версия \EmptyLine \\
Утилиты операционной системы \EmptyLine \\ \EmptyLine \\
Прикладные системы и программы \EmptyLine \\
Местонахождение и имена файлов программ и данных \EmptyLine \\ \EmptyLine \\

%=========================================================
%====================ОБЕСПЕЧЕНИЕ==========================
%=========================================================
\textit{Программное обеспечение ЭВМ студента, если использовалось:}\\
Операционная система семейства \arulefill{UNIX}\EmptyLine, наименование \arulefill{Ubuntu}\EmptyLine \ версия \arulefill{22.04}\EmptyLine \\
интерпретатор команд \arulefill{GNU bash}\EmptyLine \ версия \arulefill{5.1.16}\EmptyLine \\
Система программирования \arulefill{ Visual Studio Code}\EmptyLine \ версия \arulefill{ 1.77.3}\EmptyLine \\
Редактор текстов \arulefill{Sublime Text 3}\EmptyLine \ версия \arulefill{3211}\EmptyLine \\
Утилиты операционной системы \arulefill{Стандартные утилиты OS Linux}\EmptyLine \\
Прикладные системы и программы \arulefill{Редактор текста nano.}\EmptyLine \\
Местонахождение и имена файлов программ и данных на домашнем компьютере \\\arulefill{/home/bulat/Studying/prprm/cr/VII/}\EmptyLine \\
%=========================================================

\item 
\begin{minipage}[t][0.45\textheight]{.95\textwidth}
\textbf{Идея, метод, алгоритм} {\footnotesize решение задачи (в формах: словесной, псевдокода, графической [блок-схема, диаграмма, рисунок, таблица] или формальные спецификации с пред- и постусловиями)} \\ \Textvi

{\footnotesize Для хранения разреженной матрицы будет использована структура из одностороннего списка, искомого максимального по модулю элемента и соответствующего ему столбца. \\
Поля этой струкуры будут определяться во время считывания матрицы. \\
Делиться элементы соответствующего столбца будут в последовательности, в которой они внесены в список.} 
\end{minipage}
\item 
\begin{minipage}[t][0.43\textheight]{.95\textwidth}
\textbf{Сценарий выполнения работы} {\footnotesize (план работы, первоначальный текст программы в черновике [можно на отдельном листе] и тесты либо соображения по тестированию)} \\ \Textvii

{\footnotesize Составить makefile, написать реализацию линейного одностороннего списка и и функции считывания матрицы, выполнения соответствующих вычислений и печати разреженной матрицы во внешнем и внутреннем представлениях.}

\end{minipage}

\textit{Пункты 1-7 отчета составляются строго до начала лабораторной работы.} \\
\hfill\hbox{\textit{Допущен к выполнению работы.} \textbf{Подпись преподавателя \mrule{4cm}}}

\item 
\textbf{Распечатка протокола} {\footnotesize (подклеить листинг окончательного варианта программы с тестовыми примерами, подписанный преподавателем)} \\

%=========================================================
%====================ПРОТОКОЛ=============================
%=========================================================

\scriptsize
\centering
\begin{verbatim}
    bulat@bulat-Swift-SF314-58:~/Studying/prprm/cr/VII$ ls
    list.c  list.h  logs  main.c  makefile  test1  test2  test3
    bulat@bulat-Swift-SF314-58:~/Studying/prprm/cr/VII$ cat makefile 
    CC = gcc
    CFLAGS = -std=c99 -Wall -Wextra
    
    main: main.o list.o
        $(CC) $(CFLAGS) -o main main.o list.o
    main_o:
        $(CC) $(CFLAGS) -c main.c
    deque_o:
        $(CC) $(CFLAGS) -c list.c
    clean:
        rm -f *.o main
    bulat@bulat-Swift-SF314-58:~/Studying/prprm/cr/VII$ cat list.h
    #ifndef _LIST_H
    #define _LIST_H
    
    #include <stdbool.h>
    #include <inttypes.h>
    #include <stdlib.h>
    #include <stdio.h>
    
    typedef uint64_t size_t;
    
    typedef struct {
        char* p;
        size_t l;
    } string;
    
    string make_string(char* const);
    bool strings_eq(string, string);
    
    typedef struct node {
        struct node* r;
        double x;
    } node;
    
    node* make_node(double);
    
    void link_node(node*, node*, node*);
    
    void link_nodes(node*, node*);
    
    void node_free(node*);
    
    typedef struct {
        node* begin;
        node* back;
        size_t size;
    } list;
    
    void list_create_empty(list*);
    
    list list_copy(list* const);
    
    void list_add(list*, double, size_t);
    
    node* list_find_node(list* const, double);
    
    void list_remove(list*, double);
    
    void list_remove_subtree(list*, node*, node*);
    
    void list_free(list*);
    
    size_t list_len(node* const, node* const);
    
    void list_print(list* const);
    
    void list_cut(list*, char, double);
    
    
    #define max(a, b) ((a) > (b) ? (a) : (b))
    #define min(a, b) ((a) < (b) ? (a) : (b))
    #define abs(x) ((x) > 0 ? (x) : (-x))
    
    #endifbulat@bulat-Swift-SF314-58:~/Studying/prprm/cr/VII$ cat list.c
    #include "list.h"
    
    string make_string(char* const p) {
        string s;
        s.p = p;
        s.l = 0;
        while (s.p[s.l] != '\0') {
            ++s.l;
        }
        return s;
    }
    
    bool string_eq(string a, string b) {
        if (a.l != b.l) {
            return false;
        }
        size_t i = 0;
        while (i < a.l) {
            if (a.p[i] != b.p[i]) {
                return false;
            }
            ++i;
        }
        return true;
    }
    
    node* make_node(double x) {
        node* p = (node*)malloc(sizeof(node));
        p->r = NULL;
        p->x = x;
        return p;
    }
    
    void link_node(node* l, node* m, node* r) {
        if (! (l == NULL))
            l->r = m;
        m->r = r;
    }
    
    void link_nodes(node* l, node* r) {
        if (! (l == NULL)) {
            l->r = r;
        }
    }
    
    void node_free(node* p) {
        p->r = NULL;
        p->x = 0;
    }
    
    void list_create_empty(list* l) {
        l->begin = l->back = NULL;
        l->size = 0;
    }
    
    list list_copy(list* const l) {
        list _l;
        if (l->size == 0) {
            return _l;
        }
        _l.begin = make_node(l->begin->x);
        node* _p = _l.begin;
        node* p = l->begin->r;
        while (true) {
            _p->r = make_node(p->x);
            p = p->r;
            _p = _p->r;
            if (p == NULL) {
                break;
            }
        }
        _l.back = _p;
        _l.size = l->size;
        return _l;
    }
    
    void list_add(list* l, double x, size_t index) {
        node* _p = make_node(x);
        if (index < l->size) {
            node* p = l->begin;
            for (size_t i = 0; i < index; ++i) {
                p = p->r;
            }
            link_node(p, _p, p->r);
        } else {
            if (l->back == NULL) {
                l->begin = l->back = _p;
                l->size = 0;
            } else {
                link_node(l->back, _p, NULL);
                l->back = _p;
            }
        }
        l->size++;
    }
    
    node* list_find_node(list* const l, double x) {
        node* p = l->begin;
        while (! (p == NULL) && p->x != x) {
            p = p->r;
        }
        return p;
    }
    
    void list_remove(list* l, double x) {
        if (! (l->begin == NULL) && l->begin->x == x) {
            node* r = l->begin;
            l->begin = l->begin->r;
            if (l->begin == NULL) {
                l->back = NULL;
            } else {
                link_nodes(l->begin, l->begin->r->r);
            }
            node_free(r);
            l->size--;
        }
        else {
            node* p = l->begin;
            while (!(p->r == NULL) && p->r->x != x) {
                p = p->r;
            }
            if (! (p->r == NULL)) {
                node* r = p->r;
                link_nodes(p, p->r->r);
                node_free(r);
                l->size--;
            }
        }
    }
    
    void list_remove_subtree(list* l, node* s, node* e) {
        node* p = l->begin,* r;
        if (p != s) {
            while (! (p->r == NULL) && ! (p->r == s)) {
                p = p->r;
            }
            if (p->r == NULL) {
                return;
            }
            if (e == NULL) {
                l->back = p;
            }
            r = p->r;
            p->r = e;
            p = r;
        } else {
            l->begin = e;
        }
        while (! (p->r == NULL) && ! (p->r == e)) {
            r = p;
            p = p->r;
            node_free(r);
            --l->size;
        }
        node_free(p);
        --l->size;
    }
    
    void list_free(list* l) {
        if (l->begin == NULL) {
            l->back = NULL;
            l->size = 0;
            return;
        }
        node* p = l->begin,* r;
        while (! (p->r == NULL)) {
            r = p;
            p = p->r;
            node_free(r);
        }
        node_free(p);
        l->begin = l->back = NULL;
        l->size = 0;
    }
    
    size_t list_len(node* const l, node* const r) {
        if (l == NULL || r == NULL) {
            return 0;
        }
        node* p = l;
        size_t len = 1;
        while (! (p == r)) {
            p = p->r; ++len;
        }
        return len;
    }
    
    void list_print(list* const l) {
        printf("{");
        if (! (l->begin == NULL)) {
            node* p = l->begin;
            while (! (p->r == NULL)) {
                if (abs(p->x - (int64_t)(p->x)) < 1e-3) {
                    printf(" %ld,", (int64_t)(p->x));
                } else {
                    printf(" %.3f,", p->x);
                }
                p = p->r;
            }
            if (p->x - (int64_t)(p->x) < 1e-3) {
                printf(" %ld ", (int64_t)(p->x));
            } else {
                printf(" %.3f ", p->x);
            }
        }
        printf("}");
    }
    
    void list_cut(list* l, char side, double x) {
        node* p = list_find_node(l, x);
        if (p == NULL) {
            return;
        }
        if (side == 'l' && ! (p == l->begin)) {
            list_remove_subtree(l, l->begin, p);
        } else if (side == 'r' && ! (p->r == NULL)) {
            list_remove_subtree(l, p->r, NULL);
        }
    }bulat@bulat-Swift-SF314-58:~/Studying/prprm/cr/VII$ cat main.c 
    #include <inttypes.h>
    #include <stdbool.h>
    #include <stdio.h>
    #include <ctype.h>
    
    #include "list.h"
    
    
    typedef struct {
        list l;
        double mxe;
        size_t ind;
    } sparce_matrix;
    
    void read_sparse_matrix_from_file(sparce_matrix* mx, const char* fname) {
        FILE* f = fopen(fname, "r");
        if (f == NULL) {
            printf("Unable to open file %s", fname);
            exit(-1);
            return;
        }
    
        mx->mxe = 0;
        list* l = &(mx->l);
        list_create_empty(l);
        list_add(l, 0, l->size);
        list_add(l, 1, l->size);
        char c = '~';
        int64_t i = 1, j = 0;
    
        double x = 0;
    
        size_t fraction = 0;
        int8_t multiplier = 1;
    
        int64_t last_column = -1, penult_column = -1;
    
        while (true) {
            c = fgetc(f);
            
            if (c == ' ') {
                multiplier = 1;
                fraction = 0;
                list_add(l, j + 1, l->size);
                list_add(l, x, l->size);
                x = 0;
                ++j;
            } else if (c == '\n') {
                multiplier = 1;
                fraction = 0;
                list_add(l, j + 1, l->size);
                list_add(l, x, l->size);
                list_add(l, 0, l->size);
                list_add(l, i + 1, l->size);
                x = 0;
                ++i; j = 0;
            } else if (c == EOF) {
                list_add(l, j + 1, l->size);
                list_add(l, x, l->size);
                break;
            } else if (isdigit(c)) {
                if (fraction) {
                    x += (double)(c - 48) / fraction * multiplier;
                    fraction *= 10;
                } else {
                    x = 10 * x + (c - 48) * multiplier;
                }
            } else if (c == '.') {
                fraction = 10;
            }  else if (c == '-') {
                multiplier = -1;
            }
            if (abs(mx->mxe) < abs(x)) {
                last_column = j;
                penult_column = -1;
                mx->mxe = x;
                mx->ind = j;
            } else if (abs(abs(mx->mxe) - abs(x)) < 1e-7) {
                if (last_column < j) {
                    if (-1 < penult_column) {
                        mx->mxe = x;
                        mx->ind = last_column;
                    }
                    penult_column = last_column;
                    last_column = j;
                } else if (penult_column < j && j < last_column) {
                    penult_column = j;
                    mx->mxe = x;
                    mx->ind = penult_column;
                }
            }
        }
    
        list_add(l, 0, l->size);
        list_add(l, 0, l->size);
    
        fclose(f);
    }
    
    void print_sparse_matrix(list* l) {
        node* p = l->begin;
        if (p == NULL) {
            return;
        }
        int64_t j = 0;
        while (true) {
            if (p->x == 0) {
                if (! (p == l->begin)) {
                    printf("\n"); 
                }
                j = 0;
                if (p->r->x == 0) {
                    break;
                }
                p = p->r;
            } else {
                while (j + 1 < (int64_t)(p->x)) {
                    printf("0\t");
                    ++j;
                }
                p = p->r; ++j;
                if (abs(p->x - (int64_t)(p->x)) < 1e-3) {
                    printf(" %ld\t", (int64_t)(p->x));
                } else {
                    printf(" %.3f\t", p->x);
                }
            }
            p = p->r;
        }
    }
    
    sparce_matrix calc_func(sparce_matrix mx) {
        sparce_matrix _mx;
        _mx.l = list_copy(&(mx.l));
        _mx.mxe = mx.mxe;
        _mx.ind = mx.ind;
    
        node* p = _mx.l.begin;
        while (true) {
            if (p->x == 0) {
                if (p->r->x == 0) {
                    break;
                }
                p = p->r;
            } else {
                if (p->x == _mx.ind + 1) {
                    p = p->r;
                    p->x /= _mx.mxe;
                } else {
                    p = p->r;
                }
            }
            p = p->r;
        }
        return _mx;
    }
    
    int main(int arc, char** argv) {
    
        sparce_matrix mx;
    
        for (int i = 1; i < arc; ++i) {
            printf("File: %s\n\n", argv[i]);
    
            read_sparse_matrix_from_file(&mx, argv[i]);
    
            printf("Origin matrix:\n");
            print_sparse_matrix(&(mx.l));
            printf("Internal representation:\n");
            list_print(&(mx.l));
    
            printf("\n" );
    
            sparce_matrix _mx = calc_func(mx);
    
            printf("Calculated matrix:\n");
            print_sparse_matrix(&(_mx.l));
            printf("Internal representation:\n");
            list_print(&(_mx.l));
            printf("\n");
    
            list_free(&(mx.l));
            list_free(&(_mx.l));
            printf("\n\n");
        }
    
        
    
    
        return 0;
    }
    bulat@bulat-Swift-SF314-58:~/Studying/prprm/cr/VII$ make
    gcc -std=c99 -Wall -Wextra   -c -o main.o main.c
    gcc -std=c99 -Wall -Wextra   -c -o list.o list.c
    gcc -std=c99 -Wall -Wextra -o main main.o list.o
    bulat@bulat-Swift-SF314-58:~/Studying/prprm/cr/VII$ ./main test1 test2 test3
    File: test1
    
    Origin matrix:
     0   -5.200  0   5   -2.000 
     5.200   0   0   2   0  
     0   0   0   0   0  
     -31.000     -2.000  0   0   0  
     2   0   31  0   0  
    Internal representation:
    { 0, 1, 1, 0, 2, -5.200, 3, 0, 4, 5, 5, -2.000, 
      0, 2, 1, 5.200, 2, 0, 3, 0, 4, 2, 5, 0,
      0, 3, 1, 0, 2, 0, 3, 0, 4, 0, 5, 0, 
      0, 4, 1, -31.000, 2, -2.000, 3, 0, 4, 0, 5, 0, 
      0, 5, 1, 2, 2, 0, 3, 31, 4, 0, 5, 0, 0, 0 }
    Calculated matrix:
     0   -5.200  0   5   -2.000 
     -0.168  0   0   2   0  
     0   0   0   0   0  
     1   -2.000  0   0   0  
     -0.065  0   31  0   0  
    Internal representation:
    { 0, 1, 1, 0, 2, -5.200, 3, 0, 4, 5, 5, -2.000, 
      0, 2, 1, -0.168, 2, 0, 3, 0, 4, 2, 5, 0, 
      0, 3, 1, 0, 2, 0, 3, 0, 4, 0, 5, 0, 
      0, 4, 1, 1, 2, -2.000, 3, 0, 4, 0, 5, 0, 
      0, 5, 1, -0.065, 2, 0, 3, 31, 4, 0, 5, 0, 0, 0 }
    
    
    File: test2
    
    Origin matrix:
     3   -1.200  0  
     4   2   0  
     0   0   0  
     0   -1.000  0  
    Internal representation:
    { 0, 1, 1, 3, 2, -1.200, 3, 0, 
      0, 2, 1, 4, 2, 2, 3, 0, 
      0, 3, 1, 0, 2, 0, 3, 0, 
      0, 4, 1, 0, 2, -1.000, 3, 0, 0, 0 }
    Calculated matrix:
     0.750   -1.200  0  
     1   2   0  
     0   0   0  
     0   -1.000  0  
    Internal representation:
    { 0, 1, 1, 0.750, 2, -1.200, 3, 0, 
      0, 2, 1, 1, 2, 2, 3, 0, 
      0, 3, 1, 0, 2, 0, 3, 0, 
      0, 4, 1, 0, 2, -1.000, 3, 0, 0, 0 }
    
    
    File: test3
    
    Origin matrix:
     0   2   0  
     -2.000  0   -0.500 
     0   0.500   0  
    Internal representation:
    { 0, 1, 1, 0, 2, 2, 3, 0, 
      0, 2, 1, -2.000, 2, 0, 3, -0.500, 
      0, 3, 1, 0, 2, 0.500, 3, 0, 0, 0 }
    Calculated matrix:
     0   2   0  
     1   0   -0.500 
     0   0.500   0  
    Internal representation:
    { 0, 1, 1, 0, 2, 2, 3, 0, 
      0, 2, 1, 1, 2, 0, 3, -0.500, 
      0, 3, 1, 0, 2, 0.500, 3, 0, 0, 0 }
    
\end{verbatim}

\large{}

\newpage

\item \textbf{Дневник отладки} {\footnotesize должен содержать дату и время сеансов отладки и основные события (ошибки в сценарии и программе, нестандартные ситуации) и краткие комментарии к ним. В дневнике отладки приводятся сведения об использовании ЭВМ, существенном участии преподавателя и других лиц в написании и отладке программы.} \\
\begin{tabular}[t]{|c|c|c|c|c|c|c|}
\hline
№ & \begin{tabular}[t]{c} Лаб. \\ или \\ дом. \end{tabular} & Дата & Время & \hspace{.7cm} Событие \hspace{.7cm} & Действие по исправлению & \hspace{.7cm} Примечание \hspace{.7cm} \\
\hline
\begin{minipage}[t][0.45\textheight]{0.01\textwidth}\end{minipage} & & & & & &\\
\hline
\end{tabular}

\item \textbf{Замечания автора} {\footnotesize по существу работы:} \arulefill{\Textx} \\
\PrintEmptyLines{\value{EmptyLines2}}\\

\item \textbf{Выводы:} \arulefill{\footnotesize В ходе этой лабораторной работы я получил опыт реализации линейного одностороннего списка и работы с разреженными матрицами.} \Textxi \\
\PrintEmptyLines{\value{EmptyLines2}}\\


\item Недочёты при выполнении задания могут быть устранены следующим образом: {\footnotesize} \arulefill{\Textx}
\PrintEmptyLines{\value{EmptyLines2}}\\

 \begin{flushright}
Подпись студента \mrule{4cm}
\end{flushright}

\end{enumerate}

\end{document}