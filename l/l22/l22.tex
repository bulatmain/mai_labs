\documentclass[10pt]{article}

\usepackage{amsmath}
\usepackage{amssymb}
\usepackage[russian]{babel}
\usepackage{ragged2e}
\usepackage{fancyhdr}

% remove header line
\renewcommand{\headrulewidth}{0pt}

\setcounter{page}{8}

\usepackage[top=1.5cm, bottom=3cm, left=4cm, right=2.8cm]{geometry}

\pagestyle{fancy}
\fancyhf{}
\fancyfoot[L]{\LARGE\thepage}

\tolerance=200

\begin{document}
\Large
\begin{justify}

условие $y = 1$ при $x = 1$ ‚ окончательно имеем $xy = 1$, т.е. искомая
кривая является гиперболой.

\textit{\textbf{\largeПримечание}}. Из приведенных примеров следует также, что од-
ному и тому же дифференциальному уравнению может удовлетво-рять множество решений (при различных значениях постоянных).
Для выделения одного, определённого из них, и необходимо зада-ние дополнительных условий, что и показано в последних двух при-мерах.

Итак, \textbf{\largeобыкновенным дифференциальным уравнением} называется
соотношение вида

\[F(x,y,y^{'},y^{''},y^{'''},...,y^{(n)})=0,\eqno(1.1)\]
\\

связывающее независимую переменную $x$, функцию $y$ этой незави-
симой переменной и производные функции $y$ по $x$ до $n$-го порядка,
где функция $F$ определена и достаточное число раз дифференциру-
ема в некоторой области изменения своих аргументов.

\textbf{\largeПорядком дифференциального уравнения} называется наивысший порядок входящей в него производной.

\textbf{\largeРешением дифференциального уравнения} (1.1) называется опре-делённая и достаточное число раз дифференцируемая в некоторой
рассматриваемой области функция $y = y(x)$, в результате подстанов-
ки которой в уравнение оно становится справедливым тождеством.

Решение дифференциального уравнения, \textit{имеющее неявную фор-му} ${\textstyle \phi(x,y)=0}$ называется \textbf{\largeинтегралом дифференциального уравнения}.

\textbf{\largeРешение} дифференциального уравнения \textit{может быть определе-но} также и в \textbf{\large параметрической форме}, а именно ${\displaystyle x =\phi(t), y=\psi(t)}$. 		

\textbf{\largeПроцесс отыскания решения} дифференциального уравнения на-зывается его \textbf{\largeинтегрированием}.

\textbf{\largeГрафик решения} дифференциального уравнения называется его \textbf{\largeинтегральной кривой}.

\textbf{\largeОбщим решением} дифференциального уравнения в некоторой области его определения называется функция ${\displaystyle y=y(x,C_1,C_2,C_3,...,C_n)}$ переменной $x$, содержащая в качестве аргументов $n$ произвольных постоянных, такая, что при каждом наборе значений этих постоян-\\ных данная функция является решением дифференциального урав-нения.

Если \textit{общее решение} имеет \textit{неявный вид} ${\displaystyle \phi(x,y,C_1,C_2,C_3,...,C_n)=0}$ то такая форма общего решения называется \textbf{\large{общим интегралом}} дифференциального уравнения.

\textit{Каждое решение в составе общего решения} или \textit{каждый интег-рал, входящий в состав общего интеграла} при определённых значе-ниях постоянных, называется соответственно \textbf{\large частным решением} или
\textbf{\large частным интегралом}.

Всюду в дальнейшем, если особо не оговорено, решения диф-
ференциальных уравнений независимо от их явной или неявной
формы будем называть просто решениями.

\textit{Множество всех решений} некоторых дифференциальных урав-
нений в области их определения, кроме общего решения, \textit{может
включать} в себя дополнительно \textit{отдельные решения, не содержащие-
ся в общем решении} ни при каких значениях произвольных постоян-
ных. Такие решения также называются \textbf{\large частными решениями}.

\texit{\textbf{\largeЗамечание}}. В дальнейшем (в $\S11$ книги) будет введено \textit{определе-ние одного} из \texit{частных решений}, которое принадлежит множеству
решений уравнения и обладает определёнными специальными свой-ствами, а именно так называемое \textbf{\large особое решение}.

\textit{Графическое представление всех решений} дифференциального
уравнения составляет так называемое \textbf{\large множество интегральных кри-вых}.

Обратим внимание, что дифференциальное уравнение может и
не иметь решений в действительной области. Примером является уравнение ${\displaystyle y^{'2} + 1 = 0 }$. В основном же дифференциальные уравнения имеют множество решений. Однако можно привести пример урав-нения ${\displaystyle (y-x)^{2}+\sqrt{1-y^{'2}}=0}$, которое имеет лишь единственное решение ${\displaystyle x=y}$. Может быть также, что уравнение имеет множество решений, но эти решения не выражаются в элементарных функциях, как, например, для уравнения ${\displaystyle y^{'}=\frac{sin(x)}{x}}$ имеем ${\displaystyle y=\int{\frac{sinx}{x}dx}+C}$.
\newline
В соответствии с этим, если интегрирование дифференциального.

\end{justify}
\end{document}